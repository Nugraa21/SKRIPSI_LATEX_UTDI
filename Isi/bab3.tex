%!TEX root = ./template-skripsi.tex
% Gunakan \cite{citekey} untuk IEEE Style
% Gunakan \parencite{citekey} untuk APA Stle
%-------------------------------------------------------------------------------
%                            BAB III
%               		METODE PENELITIAN
%-------------------------------------------------------------------------------

\chapter{METODE PENELITIAN}
Bagian ini menyajikan secara lengkap setiap langkah eksperimen yang dilakukan dalam penelitian menggunakan\textbf{ bentuk kalimat pasif} yang antara lain meliputi:

\section{Bahan/Data} 
Semua bahan/data yang digunakan dikelompokkan sesuai fungsinya berdasarkan kebutuhan analitis dan teknis. Pastikan 1 alenia megandung lebih dari 2 kalimat utuh.

\section{Peralatan} 
Semua peralatan yang digunakan untuk menjalankan penelitian harus disebutkan dan diuraikan dengan jelas dan apabila perlu (terutama peralatan yang dirancang khusus) dapat disertai dengan bagan dan keterangan secukupnya. Peralatan terdiri dari hardware dan software. Hardware yang diuraikan adalah peralatan yang mendukung implementasi dan pengujian. Software merupakan perangkat lunak yang digunakan pada penelitian, dilengkapi dengan versi dan kegunaan. Untuk instrumentasi khusus merk dan tipe/spesifikasi peralatan harus dicantumkan, sedangkan kondisi pengoperasian disajikan pada bagian lain yang sesuai.

\section{Prosedur dan Pengumpulan Data} 
Pada bagian ini, variabel, prosedur, organisasi dan lokasi yang akan dipelajari serta data yang akan dikumpulkan diuraikan dengan jelas, termasuk sifat, satuan dan kisarannya. Untuk pengujian dan pengolahan data diperlukan perancangan dan pengembangan sistem.

\section{Analisis dan Rancangan Sistem} 
Pada bagian ini diuraikan analisis sistem dijelaskansecara diskriptif dan dilengkapi dengan bagan. Selain itu juga diuraikan kebutuhan sistem yang meliputi kebutuhan fungsional, kebutuhan non fungsional sistem. Rancangan sistem meliputi rancangan arsitektur sistem atau gambaran umum sistem, rancangan proses, rancangan prosedural, rancangan data, dan rancangan user interface. Bebarapa kakas yang dapat digunakan antara lain Flowchart, DAD, ERD, normalisasi,UML, dll.

	
% Baris ini digunakan untuk membantu dalam melakukan sitasi
% Karena diapit dengan comment, maka baris ini akan diabaikan
% oleh compiler LaTeX.
\begin{comment}
\bibliography{daftar-pustaka}
\end{comment}
