%!TEX root = ./template-skripsi.tex
% Gunakan \cite{citekey} untuk IEEE Style
% Gunakan \parencite{citekey} untuk APA Stle
%-------------------------------------------------------------------------------
% 								BAB I
% 							LATAR BELAKANG
%-------------------------------------------------------------------------------

\chapter{PENDAHULUAN}

\section{Latar Belakang}
Berisi argumen atau alasan berdasarkan fakta atau sumber-sumber penelitian sebelumnya (bukan opini dari penulis), sehingga perlunya dibuat penelitian ini. Uraian dimulai dengan hal yang unik, fakta, masalah dan pendapat yang mendasari penelitian. Diuraikan juga alasan-alasan teoritis dan praktis perlunya penelitian ini dilakukan dan bagaimana masalah tersebut dipecahkan.\parencite{abdillahfudholi2024}.

\section{Rumusan Masalah}
Berbekal latar belakang dan kerangka pikir, masalah yang diteliti dapat dirumuskan. Masalah yang dirumuskan harus jelas dan fokus pada kata kunci utama yang unik. Dalam merumuskan masalah, deskripsi lokasi studi terutama keunikannya sudah termasuk dalam pertimbangan. Untuk memperjelas perumusan masalah, dapat juga dibuat beberapa pertanyaan yang hendak dijawab dalam penelitian itu. Dalam uraian harus tercakup pendekatan yang digunakan dalam perumusan masalah. Untuk membantu mengikuti alur pikir secara skematis, dapat juga dibuat bagan alir kerangka proses dan rumusan masalah serta pencapaian tujuan penelitian\parencite{adha2021}.


\section{Ruang Lingkup}
Berisi uraian yang menjelaskan kompleksitas atau lingkup obyek yang diteliti. Ruang lingkup berisi deskripsi pekerjaan yang akan dilakukan (bukan apa yang tidak dikerjakan). Contohnya variabel, kriteria, ciri, cara pengujian, metode, model dan lain-lain. Disajikan dalam bentuk pointer.


\section{Tujuan Penelitian}
Menjelaskan tujuan yang akan dicapai sebagai upaya pemecahan masalah yang dijelaskan dalam latar belakang. Gunakan kata kerja yang hasilnya dapat diukur, bukan menunjukan syarat kelulusan.


\section{Manfaat Penelitian}
Menjelaskan manfaat atau kegunaan hasil penelitian bagi kepentingan pengembangan ipteks, pertimbangan dalam mengambil kebijakan, kepentingan profesi maupun masyarakat pada umumnya.

Jika ada bagian yang berbentuk enumerate maka dapat mencontoh kode ini:
\begin{enumerate}[noitemsep]
	\item ini bagian yang diberi list
	\item ini nomor yang diberi list
	\item ini nomor yang diberi list lain
\end{enumerate}

\section{Sistematika Penulisan}
Keterangan masing-masing isi bab secara ringkas. Gambaran umum tiap bab akan diterangkan pada subbab ini, dengan cara deskriptif, bukan dalam bentuk daftar. Jangan pindahkan Daftar Isi ke sini.\\
\noindent
\textbf{BAB I : PENDAHULUAN}

Pada bab ini dijelaskan latar belakang, rumusan masalah, batasan, tujuan, manfaat, keaslian penelitian, dan sistematika penulisan.\\

\noindent
\textbf{BAB II : TINJAUAN PUSTAKA DAN DASAR TEORI}

Pada bab ini dijelaskan teori-teori dan penelitian terdahulu yang digunakan sebagai acuan dan dasar dalam penelitian.\\

\noindent
\textbf{BAB III : METODE PENELITIAN}

Pada bab ini dijelaskan metode yang digunakan dalam penelitian meliputi langkah kerja, pertanyaan penilitian, alat dan bahan, serta tahapan dan alur penelitian.\\

\noindent
\textbf{BAB IV :IMPLEMENTASI DAN PEMBAHASAN}

Pada bab ini dijelaskan implementasi penelitian dan pembahasannya.\\

\noindent
\textbf{BAB V : PENUTUP }

Pada bab ini ditulis kesimpulan akhir dari penelitian dan saran untuk pengembangan penelitian selanjutnya.\\

% Baris ini digunakan untuk membantu dalam melakukan sitasi
% Karena diapit dengan comment, maka baris ini akan diabaikan
% oleh compiler LaTeX.
\begin{comment}
\bibliography{daftar-pustaka}
\end{comment}
